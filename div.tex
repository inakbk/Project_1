\documentclass[11pt,a4wide]{article}
\usepackage{verbatim}
\usepackage{listings}
\usepackage{graphicx}
\usepackage{a4wide}
\usepackage{color}
\usepackage{amsmath}
\usepackage{amssymb}
\usepackage[dvips]{epsfig}
\usepackage[T1]{fontenc}
\usepackage{cite} % [2,3,4] --> [2--4]
\usepackage{shadow}
\usepackage{hyperref}

\setcounter{tocdepth}{2}

\lstset{language=c++}
\lstset{alsolanguage=[90]Fortran}
\lstset{basicstyle=\small}
\lstset{backgroundcolor=\color{white}}
\lstset{frame=single}
\lstset{stringstyle=\ttfamily}
\lstset{keywordstyle=\color{red}\bfseries}
\lstset{commentstyle=\itshape\color{blue}}
\lstset{showspaces=false}
\lstset{showstringspaces=false}
\lstset{showtabs=false}
\lstset{breaklines}

\title{Project 1, FYS-3150}
\author{DIV}
\date{\today}
\begin{document}

\maketitle

The aim of this project is to get familiar with various vector and matrix operations,
from dynamic memory allocation to the usage of programs in the library package of the course. 

For C++ user however, there are several possible options. Two are listed here:
\begin{enumerate}
\item For this exercise we recommend that you make your own functions for dynamic memory allocation of a 
vector and a matrix. You don't need to write a class for this operations. 
Use then the 
library package lib.cpp with its header file 
lib.hpp for obtaining LU-decomposed matrices, solve linear equations
etc.
\item A very good and often recommended library for C++ handling of arrays is the library Armadillo, to be found at \url{arma.sourceforge.net}.  We will discuss the usage of this library during the lab sessions and lectures. Armadillo has also an interface to Lapack functions for solving systems of linear equations.
\end{enumerate}

Your program should include dynamic memory handling of matrices and vectors. 

The material needed for this project is covered by chapter 6 of the lecture notes, in particular section 6.4 and subsequent sections.




\end{document}